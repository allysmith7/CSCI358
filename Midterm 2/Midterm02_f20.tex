%=============================================================================
\documentclass[12pt,twoside]{article}
\usepackage{latexsym}
\usepackage{graphicx}
\usepackage{amsmath}
\usepackage{amssymb}
\usepackage{enumerate}
\usepackage{booktabs}
\usepackage{pifont}
%=============================================================================
\setlength{\evensidemargin}{+0.00in}
\setlength{\oddsidemargin} {+0.00in}
\setlength{\textwidth}     {+6.50in}
\setlength{\topmargin}     {-0.00in}
\setlength{\textheight}    {+9.00in}
\setlength{\tabcolsep}    {0.20in}
%=============================================================================
\newcommand{\xmark}{\text{\ding{55}}}
\renewcommand{\baselinestretch}{1.2}
\normalsize
%\pagestyle{plain}
\pagestyle{headings}
\markboth{CSCI358 Midterm 2}{CSCI358 Midterm 2}
\thispagestyle{empty}
%=============================================================================
\begin{document}
%
\setcounter{page}{0}
\begin{center}
	{\bf Fall 2020 CSCI358 Midterm 2}
\end{center}
%
\noindent
{\bf Instructions:} \\
%
\begin{itemize}
	%
	\item There are five questions in this paper. Please use the space provided
	(next to the questions) to write the answers, in the same way that you do for homework. You will submit your exam on Gradescope.
	%
	\item Budget your time to answer various questions and avoid spending
	too much time on a particular question.
	\item The due date for this exam is 8:00am Nov 13. You have 72 hours to work on it.
	\item Since this is an exam, you can only ask clarification questions and are not supposed to discuss solutions on Piazza.
	%

\end{itemize}
%

\renewcommand{\baselinestretch}{2.0}
\normalsize

\renewcommand{\baselinestretch}{1.2}
\normalsize
%
%
\newpage
{\bf Problem 1. (8)}
\begin{enumerate}
	\item
	Let
	\begin{align*}
		\centering
		A &= \{x|x \in \mathbb{N} \text{ and } 1 <x<50\}\\
		B &= \{x|x \in \mathbb{R} \text{ and } 1 <x<50\}\\
		C &= \{x|x \in \mathbb{Z} \text{ and } |x| \geq 25\}
	\end{align*}
	Which of the following are true? (You can put check mark ``\ding{51}'' after the true ones and cross ``\ding{55}'' after the false ones.)

	\begin{enumerate}[a.]
		\item
		$ A\subseteq B $ \ding{51}

		\item
		$17 \in A$ \ding{51}

		\item
		$A \subseteq C$ \ding{55}

		\item
		$-40 \in C$ \ding{51}

		\item
		$ \emptyset \in B$ \ding{55}

		\item
		$ \{0,1,2\} \subseteq A$ \ding{55}

		\item
		$\sqrt{3} \in B$ \ding{51}

		\item
		$\{x|x \in \mathbb{Z} \text{ and } x^2 \geq 625\} \subseteq C$ \ding{51}

		\item
		$(A -B) \cup (B - C) = A - C$ \ding{51}
		%\ding{51}
		\item
		$A \times B = B \times A$ \ding{55}
		%\ding{55}
	\end{enumerate}

	\vspace{0in}
\end{enumerate}
%
%
\newpage
{\bf Problem 2. (10 + 10 + 10)}
%
\begin{enumerate}
	\item
	Prove that
	\begin{center}
		$ A \cap (B \cup A') =   B\cap A$
	\end{center}
	where $A$ and $B$ are arbitrary sets.

	\begin{align*}
		A \cap (B \cup A') &= B \cap A\\
		(A \cap B) \cup (A \cap A') &=\\
		(A \cap B) \cup \varnothing &=\\
		A \cap B &= B \cap A
	\end{align*}
	Therefore, for any arbitrary sets $A$ and $B$, $ A \cap (B \cup A') =   B\cap A$



	\newpage
	\item
	Prove that $(A-B) \cap (B-A) = \varnothing$, where $A$ and $B$ are arbitrary sets.\\

	Proof by contradiction

	\begin{enumerate}
		\item $x \in (A - B) \to (x \in A \land x \notin B)$
		\item $(x \in A \land x \notin B) \to (x \notin B)$
		\item $(x \notin B) \to (x \notin (B - A))$
	\end{enumerate}

	Therefore, the claim must be true because the intersection between disjoint sets is $\varnothing$.

	\newpage
	\item
	Prove that $2^A \cup 2^B \subseteq 2^{A \cup B}$, where $A$ and $B$ are arbitrary sets. \\

		Assume $x \in 2^A \cap 2^B$.

		$(x \in 2^A \cap 2^B) \to (x \in 2^A \land x \in 2^B)$\\
		$(x \in 2^A \land x \in 2^B) \to (x \subseteq A \land x \subseteq B)$\\
		$(x \subseteq A \land x \subseteq B) \to (x \subseteq (A \cap B))$\\
		$(x \subseteq (A \cap B)) \to (x \in 2^{A \cap B})$

		Therefore, $2^A \cup 2^B \subseteq 2^{A \cup B}$ is true for any arbitrary sets.

\end{enumerate}
%
%
\newpage

{\bf Problem 3. (10 + 10 + 10)}
%
Prove the following properties for Fibonacci sequence:

\begin{enumerate}
	\item
	$F^2_{n+1} = F^2_{n} + F_{n-1}F_{n+2}$ for $n \geq 2$

	Basis Step ($n = 2$):

	$F_3^2 = F_2^2 + F_1F_4$

	$4 = 1^2 + (1)(3) = 4$

	Inductive Step:
	\begin{align*}
		F_{n+2} &= F^2_{n+1} + F_{n}F_{n+3}\\
		\\
		F_{n+2}^2 &= F_{n+2}F_{n+2}\\
		&= F_{n+2}\cdot(F_n + F_{n+1})\\
		&= F_{n+2}F_n + F_{n+2}F_{n+1}\\
		&= (F_{n+3} - F_{n+1})\cdot F_n + (F_{n+1} + F_{n})\cdot F_{n+1}\\
		&= F_{n+3}F_{n} - F_{n+1}F_{n}+F_{n+1}^2 +F_{n+1}F_{n}\\
		&= F_{n+1}^2 + F_{n}F_{n+3}
	\end{align*}

	Therefore, we have proven the statement to be true.

	\newpage
	\item
	$F_1 + 2F_2 + 3F_3 + \cdots + nF_{n} = nF_{n+2}-F_{n+3} +2$ for $n \geq 1$

	Basis Step ($n = 1$):

	$F_1 = 1$

	$F_1 = (1)F_3 - F_4 + 2$

	$F_1 = 2 - 3 + 2 = 1$

	Inductive Step:

	\begin{align*}
		(n+1)F_{n+3} - F_{n+4} + 2 &= F_1 + 2F_2 + 3F_3 + \cdots + nF_{n} + (n+1)F_{n+1}\\
		&= nF_{n+2}-F_{n+3} +2 + (n+1)F_{n+1}\\
		&= nF_{n+2}+F_{n+2}- F_{n+2} -F_{n+3} + (n+1)F_{n+1} +2\\
		&= (n+1)\cdot F_{n+2} - (F_{n+2} + F_{n+3}) + (n+1)F_{n+1} +2\\
		&= (n+1)\cdot F_{n+2} - F_{n+4} + (n+1)F_{n+1} +2\\
		&= (n+1)\cdot (F_{n+1}+F_{n+2}) - F_{n+4}+2\\
		&= (n+1)\cdot F_{n+3} - F_{n+4}+2\\
	\end{align*}

	Therefore, we have proven the statement to be true.

	\newpage
	\item
	$F_2 + F_4 + \cdots + F_{2n} = F_{2n+1} - 1$ for $n \geq 1$

	Basis Step ($n = 1$):

	$F_2 = F_3 - 1 = 1$

	Inductive Step:

	$F_2 + F_4 + \cdots + F_{2n} + F_{2(n+1)} = F_{2(n+1)+1} - 1$

	$F_{2n+1} - 1 + F_{2(n+1)} = F_{2(n+1)+1} - 1$

	$F_{2n+1} + F_{2n+2} - 1= F_{2n+3} -1$

	$F_{2n+1} + F_{2n+2}= F_{2n+3}$

	Therefore, we have proven the statement to be true.

\end{enumerate}
%
%
\newpage

{\bf Problem 4. (8 + 8)}
%
Give a recursive definition for each of the following set:

\begin{enumerate}
	\item
	The set of all binary strings starting with $1$.

	\begin{itemize}
		\item Basis Step: $1 \in S$
		\item Recursive Step: $(x \in S) \to (x0 \in S \land x1 \in S)$
	\end{itemize}

	\newpage
	\item
	The set of all binary strings containing no more than a single $1$.

	\begin{itemize}
		\item Basis Step: $0 \in S, 1 \in S, ``" \in S$
		\item Recursive Step: $(x \in S) \to (x0 \in S \land 0x \in S)$
	\end{itemize}

\end{enumerate}
%

%
%
\newpage

{\bf Problem 5. (8 + 8)}
%
Find the closed-form solution to each of the following recursively defined sequences:

\begin{enumerate}
	\item
	$T_1=2$ \\
	$T_n=T_{n-1}+n$ for $n \ge 2.$ \\
	(Hint: $1 + 2 + \cdots + n = \frac{n(n+1)}{2}$)

	$T_n = \frac{n(n+1)}{2} + 1$

	\vspace{3.5in}
	\item
	$S_1 = 2$ \\
	$S_n = 2S_{n-1} + n2^n$ for $n \geq 2$. \\
	(Hint: $1 + 2 + \cdots + n = \frac{n(n+1)}{2}$)

	$S_n = 2^{n-1}(n^2+n)$

\end{enumerate}
\end{document}
