%=============================================================================
\documentclass[12pt]{article}
\usepackage{latexsym}
\usepackage{graphicx}
\usepackage{booktabs}
\usepackage{multirow}
\usepackage{amsmath}
\usepackage{amssymb}
\usepackage{multicol}
\usepackage[caption=false]{subfig}


%=============================================================================
\setlength{\evensidemargin}{-0.25in}
\setlength{\oddsidemargin} {-0.25in}
\setlength{\textwidth}     {+7.00in}
\setlength{\topmargin}     {+0.00in}
\setlength{\textheight}    {+8.50in}
%=============================================================================
\makeatletter
\renewcommand{\baselinestretch}{1.2}
\normalsize
%=============================================================================

%==============================================================================
\pagestyle{plain}
%
\date{}
\begin{document}
	%==============================================================================
	\begin{flushleft}
		\large \bf
		Homework 5 \\
	\end{flushleft}
	%==============================================================================
{\bf
Please note that handwritten assignments will not be graded. To fill out your homework, use either the Latex template or the Word template (filled out in Word or another text editor). Please do not alter the order or the spacing of questions (keep them on their own pages). When you submit to Gradescope, please indicate which pages of your submitted pdf contain the answers to each question. If you have any questions about the templates or submission process, you can reach out to the TAs on Piazza. This assignment is due at 23:59 on Oct 20th.
}


\begin{enumerate}


		\item ($3 \times 5$)
		Answer the following questions about sets:
		\begin{enumerate}
			\item
			Let $S = \{2, 5, 17, 27\}$. Which of the following expressions are true?

			a. $5 \in S$ \quad\quad b. $2 + 5 \in S$ \quad\quad c. $\varnothing \in S$ \quad\quad d. $S \in S$

			\textbf{a}

			\item
			How many different sets are described  in the following? What are they?
			\begin{multicols}{2}
				$\{2, 3, 4\} $

				\hangindent=1cm $\{x \mid x$ is the first letter of cat, bat, \\or apple$\}$

				$\{x \mid x \in \mathbb{N} \text{ and } 2 \leq x \leq 4\}$

				$\{a, b, c\}$

				$\varnothing $

				\hangindent=1cm $\{x \mid x$ is the first letter of cat, bat, \\and apple$\}$

				$\{2, a, 3, b, 4, c\}$

				$\{3, 4, 2\}$
			\end{multicols}

			\textbf{4}
			\item
			Describe each of the following sets by listing its elements:

			a. $\{x \mid x \in \mathbb{N} \text{ and } x^2 -5x +6 = 0\}$

			$\{2, 3\}$

			b. $\{x \mid x \in \mathbb{R} \text{ and } x^2 = 7\}$

			$\{\pm\sqrt{7}\}$

			c. $\{x \mid x \in \mathbb{N} \text{ and } x^2 -2x -8 = 0\}$

			$\{-2, 4\}$
			\clearpage
		\end{enumerate}


		\item ($8$)	Prove that if $A \subseteq B$ and $B \subseteq C$, then $A \subseteq C$.
	%	==================================
		%  Write your answer here
		\\Proof:

		\begin{enumerate}
			\begin{enumerate}
				\item Assume $\forall x(x \in A)$.
				\item $(x \in A) \to (x \in B)$\hfill(Definition of Subset)
				\item $x \in B$\hfill(i, ii)
				\item $(x \in B) \to (x \in C)$\hfill(Definition of Subset)
				\item $x \in C$\hfill(iii, iv)
				\item $(x \in A) \to (x \in C)$
				\item $A \subseteq C$\hfill(vi, Definition of Subset)
			\end{enumerate}
		\end{enumerate}
		%===================================
		\clearpage


		\item ($2 \times 5$) Find the following power sets:
		\begin{enumerate}
			\item
			Find  $2^S$ for $S = \{a\}$.

			$\{\varnothing, \{a\}\}$

			\item
			Find  $2^S$ for $S = \{\varnothing\}$.

			$\{\varnothing\}$

		\end{enumerate}
		\clearpage



		\item ($5 \times 6$)
		Let
		\begin{align*}
		\centering
		A &= \{p, q, r, s\}\\
		B &= \{r, t, v\}\\
		C &= \{p, s, t, u\}
		\end{align*}
		be subsets of $S = \{p, q, r, s, t, u, v, w\}$. Find
		\begin{multicols}{3}
			a. $B \cap C$

			b. $A \cup C$

			c. $C^{\prime}$

			d. $A \times B$

			e. $\left(A \cup B\right) \cap C^{\prime}$
				%==================================
			%  Write your answer here
		\\	Answers:

		\begin{enumerate}
			\item $\{t\}$
			\item $\{p, q, r, s, t, u\}$
			\item $\{q, r, v, w\}$
			\item $\{(p, r), (p, t), (p, v), (q, r), (q, t), (q, v), (r, r), (r, t), (r, v), (s, r), (s, t), (s, v)\}$
			\item $\{q, r, v\}$
		\end{enumerate}
			%===================================
			\clearpage
		\end{multicols}


		\item ($4 \times 6$)
		Which of the following can be false? If the statement can be false, provide a counterexample. If it is true, you do not need to provide an example.

		a. $\left(A \cap B\right)^{\prime} = A^{\prime} \cap B^{\prime}$

		b. $A - B = \left(B - A\right)^{\prime}$

		c. $B \times A = A \times B$

		d. $\left(A - B\right) \cup \left(B - C\right) = A - C$
		%==================================
		%  Write your answer here
		\\Answers:


		Let $A = \{4, 5, 6\}, B = \{6, 7, 8\}, C = \{8, 9, 10\}$, which are all subsets of $S = \{4, 5, 6, 7, 8, 9, 10\}$
		\begin{enumerate}
			\item False, $(A \cap B)^{\prime} = \{4, 5, 7, 8, 9, 10\} \neq A^{\prime} \cap B^{\prime} = \{9, 10\}$
			\item False, $A - B = \{4, 5, 6\} \neq (B - A)^{\prime} = \{4, 5, 9, 10\}$
			\item False, $B \times A = \{(6,4), (6,5), (6,6), (7,4), (7,5), (7,6), (8,4), (8,5), (8,6)\} \neq A \times B = \{(4,6), (4,7), (4,8), (5,6), (5,7), (5,8), (6,6), (6,7), (6,8)\}$
			\item False
			\begin{enumerate}
				\item $A - B = \{4,5\}$
				\item $B - C = \{6,7\}$
				\item $(A - B) \cup (B - C) = \{4,5,6,7\}$
				\item $A - C = \{4,5,6\}$
				\item $(A - B) \cup (B - C) \neq A - C$
			\end{enumerate}
		\end{enumerate}
		%===================================
		\clearpage
		\item ($13$) Prove that if $\left(A - B\right) \cup \left(B - A\right) = A \cup B$, then $A \cap B = \varnothing$. \\

		%==================================
		%  Write your answer here
		Proof:

		Assume there is an $x$ such that $x \in A \cap B$.

		Therefore, $x \in A$ and $x \in B$ can be assumed to be true as well.

		From this, we can assume that $x \in A \cup B = (A - B) \cup (B - A)$.

		Because $x$ is in the union of the two sets, then it must be in one, maybe both.

		Therefore, $x \in (A - B)$ or $x \in (B - A)$.

		However, this cannot be true, as if $x \in (A - B)$, then $x \notin B$ must be true, which contradicts $x \in A \cap B$.

		Therefore, the statement must be true.
	\end{enumerate}
\end{document}
%==============================================================================
