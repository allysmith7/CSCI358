%=============================================================================
\documentclass[12pt,twoside]{article}
\usepackage{latexsym}
\usepackage{graphicx}
\usepackage{booktabs}
\usepackage{xcolor}
\usepackage{amsmath}
\usepackage{amsfonts}
%=============================================================================
\setlength{\evensidemargin}{+0.00in}
\setlength{\oddsidemargin} {+0.00in}
\setlength{\textwidth}     {+6.50in}
\setlength{\topmargin}     {-0.00in}
\setlength{\textheight}    {+9.00in}
\setlength{\tabcolsep}    {0.20in}
%=============================================================================
\renewcommand{\baselinestretch}{1.2}
\normalsize
%\pagestyle{plain}
\pagestyle{headings}
\markboth{CSCI358 Midterm 1}{CSCI358 Midterm 1}
\thispagestyle{empty}
%=============================================================================
\begin{document}
%
\setcounter{page}{0}
\begin{center}
{\bf Fall 2020 CSCI358 Midterm 1}
\end{center}
%
\noindent
{\bf Instructions:} \\
%
\begin{itemize}
%
\item There are four questions in this paper. Please use the space provided
      (next to the questions) to write the answers, in the same way that you do for homework. You will submit your exam on Gradescope.
%
\item Budget your time to answer various questions and avoid spending
      too much time on a particular question.
\item The due date for this exam is 8:00am Oct 16. You have 72 hours to work on it.
\item Since this is an exam, you can only ask clarification questions and are not supposed to discuss solutions on Piazza.
%

\end{itemize}
%

\renewcommand{\baselinestretch}{2.0}
\normalsize

\renewcommand{\baselinestretch}{1.2}
\normalsize

%%%%% PROBLEM 1 %%%%%%%%

\newpage
\textbf{Problem 1. (6 x 2) Write the truth tables for the following formulas:}
\begin{enumerate}
    \item
    $A  \lor  \neg B \leftrightarrow C$\\
	\\
\begin{tabular}{|c|c|c|p{1cm}|p{2cm}|p{2.5cm}|}
		\hline \rule[-2ex]{0pt}{5.5ex} A & B & C & \lnot B & A \lor \lnot B & A \lor \lnot B \leftrightarrow C \\
		\hline \rule[-2ex]{0pt}{5.5ex} F & F & F & T & T & F \\
		\hline \rule[-2ex]{0pt}{5.5ex} F & F & T & T & T & T \\
		\hline \rule[-2ex]{0pt}{5.5ex} F & T & F & F & F & T \\
		\hline \rule[-2ex]{0pt}{5.5ex} F & T & T & F & F & F \\
		\hline \rule[-2ex]{0pt}{5.5ex} T & F & F & T & T & F \\
		\hline \rule[-2ex]{0pt}{5.5ex} T & F & T & T & T & T \\
		\hline \rule[-2ex]{0pt}{5.5ex} T & T & F & F & T & F \\
		\hline \rule[-2ex]{0pt}{5.5ex} T & T & T & F & T & T \\
		\hline
	\end{tabular} \\
	\\
    \item $\ (A \lor B) \land \lnot C \rightarrow \lnot A \lor C$\\

	\begin{tabular}{|c|c|c|c|c|c|c|c|p{2.0cm}|}
		\hline A & B & C & A \lor B & \lnot C & (A \lor B) \land \lnot C & \lnot A & \lnot A \lor C & (A \lor B) \land \lnot C \to \lnot A \lor C \\
		\hline F & F & F & F & T & F & T & T & T \\
		\hline F & F & T & F & F & F & T & T & T \\
		\hline F & T & F & T & T & T & T & T & T \\
		\hline F & T & T & T & F & F & T & T & T \\
		\hline T & F & F & T & T & T & F & F & F \\
		\hline T & F & T & T & F & F & F & T & T \\
		\hline T & T & F & T & T & T & F & F & F \\
		\hline T & T & T & T & F & F & F & T & T \\
		\hline
	\end{tabular}

\end{enumerate}

%%%%%%%%%% END PROBLEM 1 %%%%%%%%%

%%%%%%%% PROBLEM 2 %%%%%%%%
\newpage
\textbf{Problem 2. (8 x 3)}
\begin{enumerate}
    \item Prove the following using a series of equivalence laws. Do not use truth tables.
    \\$\neg (A \land B) \land (A \lor \neg B) \rightarrow \neg B$ is a tautology.
	\begin{enumerate}
		\item $\lnot(A \land B) \land (A \lor \lnot B) \to \lnot B$
		\item $(\lnot A \lor \lnot B) \land (A \lor \lnot B) \to \lnot B$\hfill(DeMorgan's)
		\item $(\lnot A \land A) \lor \lnot B \to \lnot B$\hfill(Distributive)
		\item $F \lor \lnot B \to \lnot B$
		\item $\lnot B \to \lnot B$\hfill(Identity)
		\item $\lnot B \lor B$\hfill(Implication)
		\item $T$
	\end{enumerate}


    \clearpage
    \item Simplify the following formula as much as possible (resulting in an equivalent formula with as few letters as possible). Please use equivalence laws. Do not use truth tables.\\
    $\neg (A \land B) \land (A \lor \neg B)$

	$\equiv (\lnot A \lor \lnot B) \land (A \lor \lnot B)$

	$\equiv (\lnot A \lor A) \lor \lnot B$

	$\equiv F \lor \lnot B$

	$\equiv \lnot B$


\newpage



    \item
For the following statement, prove using a truth table (a) and equivalence laws (b):
\\$(A \land \neg B) \rightarrow \neg (A \rightarrow B)$ is a tautology.
\begin{enumerate}
    \item Prove the statement is a tautology using a truth table:

	\begin{tabular}{|c|c|c|c|c|c|c|}
		\hline A & B & \lnot B & A \land \lnot B & A \to B & \lnot(A \to B) & (A \land \lnot B) \to \lnot(A \to B) \\
		\hline F & F & T & F & T & F & T \\
		\hline F & T & F & F & T & F & T \\
		\hline T & F & T & T & F & T & T \\
		\hline T & T & F & F & T & F & T \\
		\hline

	\end{tabular}
    \item Prove the statement is a tautology using equivalence laws:

	\begin{enumerate}
		\item $(A \land \lnot B) \to \lnot(\lnot A \lor B)$\hfill(Implication)
		\item $(A \land \lnot B) \to (A \land \lnot B)$\hfill(DeMorgan's)
		\item $\lnot(A \land \lnot B) \lor (A \land \lnot B)$\hfill(Implication)

		Let $P$ be $A \land \lot B$

		\item $\lnot P \lor P$
		\item $T$
	\end{enumerate}
\end{enumerate}
\end{enumerate}

\newpage


%%%%%%%% PROBLEM 3 %%%%%%%%
\textbf{Problem 3 (7 x 2): Prove the following arguments using a sequence of equivalence
laws and inference rules. Do not use truth tables.}
\begin{enumerate}
    \item $(A \rightarrow (B \rightarrow C)) \land (A \lor \neg D) \land B \rightarrow (D \rightarrow C)$

	\equiv $(A \land B \rightarrow C) \land (A \lor \neg D) \land B \land D \rightarrow C$

	\begin{enumerate}
		\item $A \land B \to C$
		\item $A \lor \lnot D$
		\item $B$
		\item $D$
		\item $D \to A$\hfill(Implication Law)
		\item $A$\hfill(d, e)
		\item $A \land B$\hfill(c, f)
		\item $C$\hfill(a, g)
	\end{enumerate}
    \clearpage
    \item $\neg P \land (Q \rightarrow P ) \land (\neg Q \rightarrow R) \rightarrow R$

	\equiv $\lnot P \land (\lnot P \to \lnot Q) \land (\lnot Q \to R) \to R$

	\begin{enumerate}
		\item $\lnot P$
		\item $\lnot P \to \lnot Q$
		\item $\lnot Q \to R$
		\item $\lnot Q$\hfill(a, b)
		\item $R$\hfill(c, d)
	\end{enumerate}
\end{enumerate}
\newpage
%%%%%%%% END PROBLEM 3 %%%%%%%%

%%%%%%%% PROBLEM 4 %%%%%%%%

\textbf{Problem 4 (10 x 5): Prove the following statements:}
\begin{enumerate}
    \item For any integer $n$, if $5n+1$ is even, then n is odd.

	Let $5n + 1 = 2k$, $n \in \mathbb{Z}$, $k \in \mathbb{Z}$

	$5n = 2k - 1$

	$n = \frac{2k - 1}{5}$

	For any odd number $(2k - 1)$ divided by $5$, the result must be odd. Therefore, we have proven the theorem.

    \newpage
    \item The sum of any three consecutive integers is divisible by 3.\\

	Let $n \in \mathbb{Z}$, $k \in \mathbb{Z}$

	$\frac{n + (n + 1) + (n + 2)}{3} = k$

	$3n + 3 = 3k$

	$3(n + 1) = 3k$

	$n + 1 = k$

    \newpage
    \item The sequence $F_n$ of Fibonacci numbers is defined by $F_{n} = F_{n-2} + F_{n-1}$ for integers $n \geq 3$, with $F_1=F_2=1$. Show that $F_1^2 + F_2^2 +\cdots +F_{n}^2 =  F_nF_{n+1}$ for any positive integer $n$ \\
	Let $P(n)$ be $F_1^2 + F_2^2 + ... + F_n^2 = F_n F_{n+1}$, $f(n) = F_1^2 + F_2^2 + ... + F_n^2$

	$F_1 = 1$, $F_2 = 1$

	Base Step:

	$F_1^2 = 1 \times 1$

	$f(n) = F_1^2 + F_2^2 + ... + F_n^2$, $f(n) = F_{n+1}F_{n+2}$

	$f(n + 1) = F_1^2 + F_2^2 + ... + F_n^2 + F_{n+1}^2$, $f(n+1) = F_{n+1}F_{n+2}$

	$= f(n) + F_{n+1}^2 = F_{n+1}F_{n+2}$

	$= F_{n}F_{n+1} + F_{n+1}^2 = F_{n+1}F_{n+2}$

	$= F_{n+1}(F_n + F_{n+1}) = F_{n+1}F_{n+2}$

	$= F_{n+1}F_{n+2} = F_{n+1}F_{n+2}$

	Therefore, we have proven the theorem.

    \newpage
    \item Let $n$ be an integer. If n is not divisible by 3, then $n^2 -1$ is divisible by 3.\\
	Let $k \in \mathbb{Z}$

	Let $n = 3k + 1$

	$n^2 - 1$

	$= (3k + 1)^2 - 1$

	$= 9k^2 + 6k + 1 - 1$

	$= 3(3k^2 + 2k)$

	Let $n = 3k  + 2$

	$n^2 - 1$

	$= (3k + 2)^2 - 1$

	$= 9k^2 + 12k + 4 - 1$

	$= 3(3k^2 + 4k + 1)$

	Therefore, for any $n$ not divisible by $3$, $n^2 - 1$ is divisible by $3$.
 	\newpage

\item Prove the following using induction or strong induction

    $1^3 + 2^3 + \cdots + n^3 = \frac{n^2(n+1)^2}{4}$ for any positive integer $n$

	Let $n \in \mathbb{N}^+$

	Let $P(n)$ be $1^3 + 2^3 + ... + n^3 = \frac{{n^2}{n+1}^2}{4}$, $f(n) = 1^3 + 2^3 + ... + n^3$

	Base Step:

	$1^3 = \frac{(1^2)(1+1)^2}{4} = 1$

	$P(n)$ is $f(n) = \frac{n^2(n+1)^2}{4}$

	$P(n+1)$ is $1^3 + 2^3 + ... + n^3 + (n+1)^3 = \frac{(n+1)^2(n+2)^2}{4}$

	$f(n) + (n+1)^3 = \frac{(n+1)^2(n+2)^2}{4}$

	$\frac{n^2(n+1)^2}{4} + (n+1)^3 = \frac{(n+1)^2(n+2)^2}{4}$

	$\frac{n^4 + 2n^3 + n^2}{4} + \frac{4(n^3 + 3n^2 + 3n + 1)}{4}= \frac{(n+1)^2(n+2)^2}{4}$

	$\frac{n^4 + 6n^3 + 13n^2 + 12n + 4}{4} = \frac{(n+1)^2(n+2)^2}{4}$

	$\frac{(n+1)^2(n+2)^2}{4} = \frac{(n+1)^2(n+2)^2}{4}$

	Therefore, we have proven the theorem to be true.
\end{enumerate}





\end{document}
